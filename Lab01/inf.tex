%package list
\documentclass{article}
\usepackage[top=3cm, bottom=3cm, outer=3cm, inner=3cm]{geometry}
\usepackage{multicol}
\usepackage{graphicx}
\usepackage{url}
%\usepackage{cite}
\usepackage{hyperref}
\usepackage{array}
%\usepackage{multicol}
\newcolumntype{x}[1]{>{\centering\arraybackslash\hspace{0pt}}p{#1}}
\usepackage{natbib}
\usepackage{pdfpages}
\usepackage{multirow}
\usepackage[normalem]{ulem}
\useunder{\uline}{\ul}{}
\usepackage{svg}
\usepackage{xcolor}
\usepackage{listings}
\lstdefinestyle{ascii-tree}{
    literate={├}{|}1 {─}{--}1 {└}{+}1 
  }
\lstset{basicstyle=\ttfamily,
  showstringspaces=false,
  commentstyle=\color{red},
  keywordstyle=\color{blue}
}
%\usepackage{booktabs}
\usepackage{caption}
\usepackage{subcaption}
\usepackage{float}
\usepackage{array}

\newcolumntype{M}[1]{>{\centering\arraybackslash}m{#1}}
\newcolumntype{N}{@{}m{0pt}@{}}


%%%%%%%%%%%%%%%%%%%%%%%%%%%%%%%%%%%%%%%%%%%%%%%%%%%%%%%%%%%%%%%%%%%%%%%%%%%%
%%%%%%%%%%%%%%%%%%%%%%%%%%%%%%%%%%%%%%%%%%%%%%%%%%%%%%%%%%%%%%%%%%%%%%%%%%%%
\newcommand{\itemEmail}{eaguilartan@unsa.edu.pe}
\newcommand{\itemStudent}{Aguilar Tancayo Edwin Francisco}
\newcommand{\itemCourse}{Programación Web 2}
\newcommand{\itemCourseCode}{1702122}
\newcommand{\itemSemester}{I}
\newcommand{\itemUniversity}{Universidad Nacional de San Agustín de Arequipa}
\newcommand{\itemFaculty}{Facultad de Ingeniería de Producción y Servicios}
\newcommand{\itemDepartment}{Departamento Académico de Ingeniería de Sistemas e Informática}
\newcommand{\itemSchool}{Escuela Profesional de Ingeniería de Sistemas}
\newcommand{\itemAcademic}{2024 - A}
\newcommand{\itemInput}{Del 2 Mayo 2023}
\newcommand{\itemOutput}{Al 7 Mayo 2023}
\newcommand{\itemPracticeNumber}{01}
\newcommand{\itemTheme}{Docker}
%%%%%%%%%%%%%%%%%%%%%%%%%%%%%%%%%%%%%%%%%%%%%%%%%%%%%%%%%%%%%%%%%%%%%%%%%%%%
%%%%%%%%%%%%%%%%%%%%%%%%%%%%%%%%%%%%%%%%%%%%%%%%%%%%%%%%%%%%%%%%%%%%%%%%%%%%

\usepackage[english,spanish]{babel}
\usepackage[utf8]{inputenc}
\AtBeginDocument{\selectlanguage{spanish}}
\renewcommand{\figurename}{Figura}
\renewcommand{\refname}{Referencias}
\renewcommand{\tablename}{Tabla} %esto no funciona cuando se usa babel
\AtBeginDocument{%
	\renewcommand\tablename{Tabla}
}

\usepackage{fancyhdr}
\pagestyle{fancy}
\fancyhf{}
\setlength{\headheight}{30pt}
\renewcommand{\headrulewidth}{1pt}
\renewcommand{\footrulewidth}{1pt}
\fancyhead[L]{\raisebox{-0.2\height}{\includegraphics[width=3cm]{img/logo_episunsa.png}}}
\fancyhead[C]{\fontsize{7}{7}\selectfont	\itemUniversity \\ \itemFaculty \\ \itemDepartment \\ \itemSchool \\ \textbf{\itemCourse}}
\fancyhead[R]{\raisebox{-0.2\height}{\includegraphics[width=1.2cm]{img/logo_abet}}}
\fancyfoot[L]{Estudiante Edwin Aguilar}
\fancyfoot[C]{\itemCourse}
\fancyfoot[R]{Página \thepage}

% para el codigo fuente
\usepackage{listings}
\usepackage{color, colortbl}
\definecolor{dkgreen}{rgb}{0,0.6,0}
\definecolor{gray}{rgb}{0.5,0.5,0.5}
\definecolor{mauve}{rgb}{0.58,0,0.82}
\definecolor{codebackground}{rgb}{0.95, 0.95, 0.92}
\definecolor{tablebackground}{rgb}{0.8, 0, 0}

\lstset{frame=tb,
	language=bash,
	aboveskip=3mm,
	belowskip=3mm,
	showstringspaces=false,
	columns=flexible,
	basicstyle={\small\ttfamily},
	numbers=none,
	numberstyle=\tiny\color{gray},
	keywordstyle=\color{blue},
	commentstyle=\color{dkgreen},
	stringstyle=\color{mauve},
	breaklines=true,
	breakatwhitespace=true,
	tabsize=3,
	backgroundcolor= \color{codebackground},
}

\begin{document}
	
	\vspace*{10px}
	
	\begin{center}	
		\fontsize{17}{17} \textbf{ Informe de Laboratorio \itemPracticeNumber}
	\end{center}
	\centerline{\textbf{\Large Tema: \itemTheme}}
	%\vspace*{0.5cm}	

	\begin{flushright}
		\begin{tabular}{|M{2.5cm}|N|}
			\hline 
			\rowcolor{tablebackground}
			\color{white} \textbf{Nota}  \\
			\hline 
			     \\[30pt]
			\hline 			
		\end{tabular}
	\end{flushright}	

	\begin{table}[H]
		\begin{tabular}{|x{4.7cm}|x{4.8cm}|x{4.8cm}|}
			\hline 
			\rowcolor{tablebackground}
			\color{white} \textbf{Estudiante} & \color{white}\textbf{Escuela}  & \color{white}\textbf{Asignatura}   \\
			\hline 
			{\itemStudent \par \itemEmail} & \itemSchool & {\itemCourse \par Semestre: \itemSemester \par Código: \itemCourseCode}     \\
			\hline 			
		\end{tabular}
	\end{table}		
	
	\begin{table}[H]
		\begin{tabular}{|x{4.7cm}|x{4.8cm}|x{4.8cm}|}
			\hline 
			\rowcolor{tablebackground}
			\color{white}\textbf{Laboratorio} & \color{white}\textbf{Tema}  & \color{white}\textbf{Duración}   \\
			\hline 
			\itemPracticeNumber & \itemTheme & 05 horas   \\
			\hline 
		\end{tabular}
	\end{table}
	
	\begin{table}[H]
		\begin{tabular}{|x{4.7cm}|x{4.8cm}|x{4.8cm}|}
			\hline 
			\rowcolor{tablebackground}
			\color{white}\textbf{Semestre académico} & \color{white}\textbf{Fecha de inicio}  & \color{white}\textbf{Fecha de entrega}   \\
			\hline 
			\itemAcademic & \itemInput &  \itemOutput  \\
			\hline 
		\end{tabular}
	\end{table}
	
	\section{Tarea}
	\begin{itemize}		



		\item  Crear un contenedor en Docker basado en ubuntu 20.04:
        \item Instale el servidor web Apache HTTP server 2.x
 \item Instale cualquiera de estos lenguajes de programación: PHP, Perl, Python.
 \item Configure el servidor web para que interprete uno de los lenguajes de programación.
 \item Instale cualquiera de los servidores de base de datos: MySQL, MariaDB, PostgreSQL.
 \item Instale el servidor Open SSH Server. Envíe archivos al servidor: imágenes, css, js, etc.
 \item Cree un usuario pw2 con contraseña: 12345678.
 \item Otorgue permisos al usuario para acceder a la aplicación web. (Read/Write). Utilizar Virtual Host.
 \item Finalmente implemente el trabajo final del curso de pw1 en ese contenedor.
  \item Elabore un informe paso a paso para donde explique funcionalmente el proyecto demostrando que se trata de un contenedor docker.
 \item Adjunte la URL de un video donde muestre que se trata de un contenedor Docker.

Entregables:
Informe de laboratorio
URL: Video Youtube.
INVESTIGACIÓN: Suba el proyecto a Docker Hub (Público).
URL: Docker Hub

  
		
	\section{Equipos, materiales y temas utilizados}
	\begin{itemize}
		\item Sistema Operativo Debian 12-xfce
		\item NEOVIM 0.10.0
		\item Docker.io
        \item Apache HTTP Server
		\item Git 2.34.1
		\item Cuenta en GitHub con el correo institucional.
	\end{itemize}
	
	\section{URL de Repositorio Github}
	\begin{itemize}
		\item URL del Repositorio GitHub para clonar o recuperar.
		\item \url{https://github.com/EdwinAguilarT/pw2-lab-d-24a.git}
		\item URL para el laboratorio 01 en el Repositorio GitHub.
		\item \url{https://github.com/EdwinAguilarT/pw2-lab-d-24a/tree/main/Lab01}
	\end{itemize}
 \section{URL del video}
	\begin{itemize}
		\item URL del video de demostracion.
		\item \url{https://drive.google.com/file/d/1WX5DuqURXzN5oiUAxKiIB65i5UC9p_uz/view?usp=drive_link}
		
	\end{itemize}

       \section{URL de Repositorio Dockerhub}
	\begin{itemize}
		\item docker push edwinaguilar/pw2_lab01:tagname

		\item \url{https://hub.docker.com/repository/docker/edwinaguilar/pw2_lab01/general}
		\item Docker commands.
		\item Docker commands.

	\end{itemize}
	
	\section{Docker.io}
	
        \subsection{Proyecto Docker.io con servidor Apache2                                
        }
	\begin{itemize}
		\item En esta sección pondré capturas del proceso.
	\end{itemize}	
        \begin{figure}[H]
  \centering
  \includegraphics[width=1\textwidth]{img/Pictures/instalarDocker.png}
  \caption{instalando docker en Debian}
  \label{fig:imagen1}
\end{figure}
  \begin{figure}[H]
  \centering
  \includegraphics[width=1\textwidth]{img/Pictures/dockerVersion.png}
  \caption{comando docker version}
  \label{fig:imagen1}
\end{figure}
  \begin{figure}[H]
  \centering
  \includegraphics[width=1\textwidth]{img/Pictures/dockerInfo.png}
  \caption{Comando docker info}
  \label{fig:imagen1}
\end{figure}
        \begin{figure}[H]
  \centering
  \includegraphics[width=1\textwidth]{img/Pictures/creandoContenedor.png}
  \caption{Creando contenedor en docker}
  \label{fig:imagen2}
\end{figure}
        \begin{figure}[H]
  \centering
  \includegraphics[width=1\textwidth]{img/Pictures/iniciar.png}
  \caption{iniciando contenedor}
  \label{fig:imagen3}
\end{figure}
        \begin{figure}[H]
  \centering
  \includegraphics[width=1\textwidth]{img/Pictures/installApache2.png}
  \caption{Instando apache2 en Docker}
  \label{fig:imagen4}
\end{figure}
        \begin{figure}[H]
  \centering
  \includegraphics[width=1\textwidth]{img/Pictures/apache2Start.png}
  \caption{Iniciando apache2}
  \end{figure}

  \label{fig:imagen5}
   \begin{figure}[H]
  \centering
  \includegraphics[width=1\textwidth]{img/Pictures/indexApache2.png}
  \caption{index.html de Apache2}
  begin{figure}[H]
  \centering
  \includegraphics[width=1\textwidth]{img/Pictures/virtualhost.png}
  \caption{virtualhost webapp01}
  \label{fig:imagen6}
  \end{figure}

  
  
  begin{figure}[H]
  \centering
  \includegraphics[width=1\textwidth]{img/Pictures/treeWebapp.png}
  \caption{Creando carpetas para el proyecto}

    \begin{figure}[H]
  \centering
  \includegraphics[width=1\textwidth]{img/Pictures/fakewiki.png}
  \caption{Mostrando proyecto}
  \end{figure}

    \begin{figure}[H]
  \centering
  \includegraphics[width=1\textwidth]{img/Pictures/fakewiki2.png}
  \caption{Mostrando Proyecto 2}
  \end{figure}
    \begin{figure}[H]
  \centering
  \includegraphics[width=1\textwidth]{img/Pictures/dockerHub.png}
  \caption{subiendo imagen a dockerhub}
  \end{figure}

   
\subsection{Datos Necesarios para el Proyecto}
	\begin{itemize}
		\item Usuario Administrador: 
        \item usuario: eaguilartan  
        \item contraseña: 12345678
        \item Usuario Administrador 2: 
        \item usuario: admin
        \item contraseña:   1234
	\end{itemize}	
	
    
 
	\section{\textcolor{red}{Rúbricas}}
	
	\subsection{\textcolor{red}{Entregable Informe}}
	\begin{table}[H]
		\caption{Tipo de Informe}
		\setlength{\tabcolsep}{0.5em} % for the horizontal padding
		{\renewcommand{\arraystretch}{1.5}% for the vertical padding
		\begin{tabular}{|p{3cm}|p{12cm}|}
			\hline
			\multicolumn{2}{|c|}{\textbf{\textcolor{red}{Informe}}}  \\
			\hline 
			\textbf{\textcolor{red}{Latex}} & \textcolor{blue}{El informe está en formato PDF desde Latex,  con un formato limpio (buena presentación) y facil de leer.}   \\ 
			\hline 
			
			
		\end{tabular}
	}
	\end{table}
	
	\clearpage
	
	\subsection{\textcolor{red}{Rúbrica para el contenido del Informe y demostración}}
	\begin{itemize}			
		\item El alumno debe marcar o dejar en blanco en celdas de la columna \textbf{Checklist} si cumplio con el ítem correspondiente.
		\item Si un alumno supera la fecha de entrega,  su calificación será sobre la nota mínima aprobada, siempre y cuando cumpla con todos lo items.
		\item El alumno debe autocalificarse en la columna \textbf{Estudiante} de acuerdo a la siguiente tabla:
	
		\begin{table}[ht]
			\caption{Niveles de desempeño}
			\begin{center}
			\begin{tabular}{ccccc}
    			\hline
    			 & \multicolumn{4}{c}{Nivel}\\
    			\cline{1-5}
    			\textbf{Puntos} & Insatisfactorio 25\%& En Proceso 50\% & Satisfactorio 75\% & Sobresaliente 100\%\\
    			\textbf{2.0}&0.5&1.0&1.5&2.0\\
    			\textbf{4.0}&1.0&2.0&3.0&4.0\\
    		\hline
			\end{tabular}
		\end{center}
	\end{table}	
	
	\end{itemize}
	
	\begin{table}[H]
		\caption{Rúbrica para contenido del Informe y demostración}
		\setlength{\tabcolsep}{0.5em} % for the horizontal padding
		{\renewcommand{\arraystretch}{1.5}% for the vertical padding
		%\begin{center}
		\begin{tabular}{|p{2.7cm}|p{7cm}|x{1.3cm}|p{1.2cm}|p{1.5cm}|p{1.1cm}|}
			\hline
    		\multicolumn{2}{|c|}{Contenido y demostración} & Puntos & Checklist & Estudiante & Profesor\\
			\hline
			\textbf{1. GitHub} & Hay enlace URL activo del directorio para el  laboratorio hacia su repositorio GitHub con código fuente terminado y fácil de revisar. &2 &X &2 & \\ 
			\hline
			\textbf{2. Commits} &  Hay capturas de pantalla de los commits más importantes con sus explicaciones detalladas. (El profesor puede preguntar para refrendar calificación). &4 &X &4 & \\ 
			\hline 
			\textbf{3. Código fuente} &  Hay porciones de código fuente importantes con numeración y explicaciones detalladas de sus funciones. &2 &X &1 & \\ 
			\hline 
			\textbf{4. Ejecución} & Se incluyen ejecuciones/pruebas del código fuente  explicadas gradualmente. &2 &X &2 & \\ 
			\hline			
			\textbf{5. Pregunta} & Se responde con completitud a la pregunta formulada en la tarea.  (El profesor puede preguntar para refrendar calificación).  &2 &X &2 & \\ 
			\hline	
			\textbf{6. Fechas} & Las fechas de modificación del código fuente estan dentro de los plazos de fecha de entrega establecidos. &2 &X &2 & \\ 
			\hline 
			\textbf{7. Ortografía} & El documento no muestra errores ortográficos. &2 &X &2 & \\ 
			\hline 
			\textbf{8. Madurez} & El Informe muestra de manera general una evolución de la madurez del código fuente,  explicaciones puntuales pero precisas y un acabado impecable.   (El profesor puede preguntar para refrendar calificación).  &4 &X &3 & \\ 
			\hline
			\multicolumn{2}{|c|}{\textbf{Total}} &20 & &18 & \\ 
			\hline
		\end{tabular}
		%\end{center}
		%\label{tab:multicol}
		}
	\end{table}
	
\clearpage

\section{Referencias}
\begin{itemize}			
	\item \url{https://www.learnlatex.org/es/}
\end{itemize}	
	
%\clearpage
%\bibliographystyle{apalike}
%\bibliographystyle{IEEEtranN}
%\bibliography{bibliography}
			
\end{document}
